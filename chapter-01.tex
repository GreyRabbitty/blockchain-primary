\section{Usage}
The term ``\textbf{\textit{blockchain}}'' on its own is ambiguous.
We might discuss \textbf{\textit{a}} blockchain such as Ethereum.
Or we could identify Bitcoin as \textbf{\textit{the}} blockchain.
A discussion of \textbf{\textit{blockchain technology}} might also be interesting as would a debate about \textbf{\textit{blockchains}}.
But we would never discuss \textbf{\textit{blockchain}}.
The term \textbf{\textit{blockchain}} on its own is not specific.

It is a bit like how we use the term ``\textbf{\textit{rocket}}''.
We might discuss \textbf{\textit{a}} rocket such as the Falcon Heavy from SpaceX.
Or we could identify NASA's Saturn V as \textbf{\textit{the}} rocket.
A discussion of \textbf{\textit{rocket technology}} might also be interesting as would a debate about \textbf{\textit{rockets}}.
But we would never discuss \textbf{\textit{rocket}}.
Likewise, the term \textbf{\textit{rocket}} on its own is not specific.

Our use of the term \textbf{\textit{blockchain}} will not be ambiguous.
\textbf{\textit{The}} blockchain is Bitcoin.
Some \textbf{\textit{blockchains}}, such as Ethereum, support generic smart contracts.
But asking \textbf{\textit{``What about Blockchain?''}} suggests an unfamiliarity we seek to resolve with this book.

\section{Definition}
The term \textbf{\textit{blockchain}} doesn't have a universal definition.

Data structure
Distributed ledger
Byzantine consensus

In this book we will use the term \textbf{\textit{blockchain}} as shorthand for \textbf{\textit{public permissionless blockchain}}.
